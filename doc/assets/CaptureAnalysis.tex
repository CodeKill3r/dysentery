\documentclass[11pt]{article}

\usepackage{bytefield}
\title{DJ Link Packet Analysis}
\author{James Elliott}

\begin{document}

\maketitle

\section{Mixer Startup}

When the mixer starts up, after it obtains an IP address (or gives up
on doing that and self-assigns an address), it sends out what look
like a series of packets simply announcing its existence to UDP port
50000 on the broadcast address of the local network.

\begin{figure}
  \begin{bytefield}[bitwidth=1.5em]{16}
    \bitheader{0-15} \\
    \begin{rightwordgroup}{Name}
      \begin{leftwordgroup}{Header}
        \bitboxes*{1}{{51} {73} {70} {74} {31} {57} {6d} {4a} {4f} {4c} {\textbf{0a}} {00}}
        & \bitbox[lrt]{4}{}
      \end{leftwordgroup} \\
      \wordbox[lrb]{1}{Device Name (padded with 00)} 
    \end{rightwordgroup} \\
    \bitboxes*{1}{{01} {02} {00} {25} {02}} \\
  \end{bytefield}
  \caption{Initial announcement packets from Mixer}
  \label{fig:mixerInitial}
\end{figure}

These have a data length of 37 bytes, appear roughly every 300
milliseconds, and have the content shown in Figure
\ref{fig:mixerInitial}.

The tenth byte (inside what is labeled the header) is bolded because
its value changes in the different types of packets which follow.

After about three of these packets are sent, another series of three
begins. It is not clear what purpose these packets serve, because they
are not yet asserting ownership of any device number; perhaps they are
used when CDJs are powering up as part of the mechanism the mixer can
use to tell them which device number to use based on which network
port they are connected to?

\begin{figure}
  \begin{bytefield}[bitwidth=1.5em]{16}
    \bitheader{0-15} \\
    \begin{rightwordgroup}{Name}
      \begin{leftwordgroup}{Header}
        \bitboxes*{1}{{51} {73} {70} {74} {31} {57} {6d} {4a} {4f} {4c} {\textbf{00}} {00}}
        & \bitbox[lrt]{4}{}
      \end{leftwordgroup} \\
      \wordbox[lrb]{1}{Device Name (padded with 00)} 
    \end{rightwordgroup} \\
    \bitboxes*{1}{{01} {02} {00} {2c} {\emph{N}} {02}} &
    \bitbox{6}{MAC address} \\
  \end{bytefield}
  \caption{First-stage Mixer device number assignment packets}
  \label{fig:mixerStage1}
\end{figure}

In any case, these three packets have a data length of 44 bytes, are
again sent to UDP port 50000 on the local network broadcast address,
at roughly 300 millisecond intervals, and have the content shown in
Figure \ref{fig:mixerStage1}.

The value \emph{N} at byte 36 is 1, 2, or 3, depending on whether this
is the first, second, or third time the packet is sent.

After these comes another series of three numbered packets. These
appear to be claiming the device number for a particular device, as
well as announcing the IP address at which it can be found. They have
a data length of 50 bytes, and are again sent to UDP port 50000 on the
local network broadcast address, at roughly 300 millisecond intervals,
with the content shown in Figure \ref{fig:mixerStage2}.

I identify these as claiming/identifying the device number because the
value \emph{D} at byte 46 is the same as the device number that the
mixer uses to identify itself (0x21) and the same is true for the
corresponding packets seen from the CDJs (they use device numbers 2
and 3, as they are connected to those ports/channels on the mixer).

\begin{figure}[h]
  \begin{bytefield}[bitwidth=1.5em]{16}
    \bitheader{0-15} \\
    \begin{rightwordgroup}{Name}
      \begin{leftwordgroup}{Header}
        \bitboxes*{1}{{51} {73} {70} {74} {31} {57} {6d} {4a} {4f} {4c} {\textbf{02}} {00}}
        & \bitbox[lrt]{4}{}
      \end{leftwordgroup} \\
      \wordbox[lrb]{1}{Device Name (padded with 00)} 
    \end{rightwordgroup} \\
    \bitboxes*{1}{{01} {02} {00} {32}} &
    \bitbox{4}{IP address} & \bitbox{6}{MAC address} &
    \bitbox{1}{\emph{D}} & \bitbox{1}{\emph{N}} \\
    \bitboxes*{1}{{02} {01}} \\
  \end{bytefield}
  \caption{Second-stage Mixer device number assignment packets}
  \label{fig:mixerStage2}
\end{figure}

As with the previous series of three packets, the value \emph{N} at
byte 47 takes on the values 1, 2, and 3 in the three packets.

These are followed by another three packets, perhaps the last stage of
claiming the device number, again at 300 millisecond intervals, to the
same port 50000. These shorter packets have 38 bytes of data and the
content shown in Figure \ref{fig:mixerStage3}.

\begin{figure}
  \begin{bytefield}[bitwidth=1.5em]{16}
    \bitheader{0-15} \\
    \begin{rightwordgroup}{Name}
      \begin{leftwordgroup}{Header}
        \bitboxes*{1}{{51} {73} {70} {74} {31} {57} {6d} {4a} {4f} {4c} {\textbf{04}} {00}}
        & \bitbox[lrt]{4}{}
      \end{leftwordgroup} \\
      \wordbox[lrb]{1}{Device Name (padded with 00)} 
    \end{rightwordgroup} \\
    \bitboxes*{1}{{01} {02} {00} {26}} &
    \bitbox{1}{\emph{D}} & \bitbox{1}{\emph{N}} \\
  \end{bytefield}
  \caption{Final-stage Mixer device number assignment packets}
  \label{fig:mixerStage3}
\end{figure}

As before the value \emph{D} at byte 36 is the same as the device
number that the mixer uses to identify itself (0x21) and \emph{N} at
byte 37 takes on the values 1, 2, and 3 in the three packets.

Once those are sent, the mixer seems to settle down and send what
looks like a keep-alive packet to retain presence on the network and
ownership of its device number, at a less frequent interval. These
packets are 54 bytes long, again sent to port 50000 on the local
network broadcast address, roughly every second and a half. They have
the content shown in Figure \ref{fig:mixerKeepalive}.

\begin{figure}[h]
  \begin{bytefield}[bitwidth=1.5em]{16}
    \bitheader{0-15} \\
    \begin{rightwordgroup}{Name}
      \begin{leftwordgroup}{Header}
        \bitboxes*{1}{{51} {73} {70} {74} {31} {57} {6d} {4a} {4f} {4c} {\textbf{06}} {00}}
        & \bitbox[lrt]{4}{}
      \end{leftwordgroup} \\
      \wordbox[lrb]{1}{Device Name (padded with 00)} 
    \end{rightwordgroup} \\
    \bitboxes*{1}{{01} {02} {00} {36}} &
    \bitbox{1}{\emph{D}} & \bitbox{1}{02} &
    \bitbox{6}{MAC address} & \bitbox{4}{IP address} \\
    \bitboxes*{1}{{01} {00} {00} {00} {02} {00}} \\
  \end{bytefield}
  \caption{Mixer keep-alive packets}
  \label{fig:mixerKeepalive}
\end{figure}

\section{CDJ Startup}

When a CDJ starts up the procedure and packets are nearly identical,
with groups of three packets sent at 300 millisecond intervals to port
50000 of the local network broadcast address. The only difference
between Figure \ref{fig:mixerInitial} and Figure \ref{fig:cdjInitial}
is the final byte, which is 01 for the CDJ, and was 02 for the mixer.

\begin{figure}[h]
  \begin{bytefield}[bitwidth=1.5em]{16}
    \bitheader{0-15} \\
    \begin{rightwordgroup}{Name}
      \begin{leftwordgroup}{Header}
        \bitboxes*{1}{{51} {73} {70} {74} {31} {57} {6d} {4a} {4f} {4c} {\textbf{0a}} {00}}
        & \bitbox[lrt]{4}{}
      \end{leftwordgroup} \\
      \wordbox[lrb]{1}{Device Name (padded with 00)} 
    \end{rightwordgroup} \\
    \bitboxes*{1}{{01} {02} {00} {25} {01}} \\
  \end{bytefield}
  \caption{Initial announcement packets from CDJ}
  \label{fig:cdjInitial}
\end{figure}

Similarly, the next series of three packets from the CDJ are nearly
identical to those from the mixer. The only differences between Figure
\ref{fig:mixerStage1} and Figure \ref{fig:cdjStage1} is byte 37
(immediately after the packet counter \emph{N}), which again is 01 for
the CDJ, and was 02 for the mixer.

\begin{figure}
  \begin{bytefield}[bitwidth=1.5em]{16}
    \bitheader{0-15} \\
    \begin{rightwordgroup}{Name}
      \begin{leftwordgroup}{Header}
        \bitboxes*{1}{{51} {73} {70} {74} {31} {57} {6d} {4a} {4f} {4c} {\textbf{00}} {00}}
        & \bitbox[lrt]{4}{}
      \end{leftwordgroup} \\
      \wordbox[lrb]{1}{Device Name (padded with 00)} 
    \end{rightwordgroup} \\
    \bitboxes*{1}{{01} {02} {00} {2c} {\emph{N}} {01}} &
    \bitbox{6}{MAC address} \\
  \end{bytefield}
  \caption{First-stage CDJ device number assignment packets}
  \label{fig:cdjStage1}
\end{figure}

However it appears that the CDJ skips the second stage of claiming a
device number, probably because it is configured to be automatically
assigned a device number based on the port of the mixer to which it is
connected, and we cannot see a packet that the mixer sent it assigning
it that device number. Instead, it jumps right to the end of the third
and final stage, sending a single 38-byte packet with header byte 10
set to 04 (which identified the three packets of the third stage when
the mixer was starting up), with content identical to Figure
\ref{fig:mixerStage3}.

Even though the value of \emph{N} is 01, this is the only packet in
this series that the CDJ sends. It would probably behave differently
if configured to assign its own device number (behaving like we saw
the mixer behave in claiming its device number).

It then moves to the final keep-alive stage, sending out 54-byte
packets with the content shown in Figure \ref{fig:cdjKeepalive}.

\begin{figure}[h]
  \begin{bytefield}[bitwidth=1.5em]{16}
    \bitheader{0-15} \\
    \begin{rightwordgroup}{Name}
      \begin{leftwordgroup}{Header}
        \bitboxes*{1}{{51} {73} {70} {74} {31} {57} {6d} {4a} {4f} {4c} {\textbf{06}} {00}}
        & \bitbox[lrt]{4}{}
      \end{leftwordgroup} \\
      \wordbox[lrb]{1}{Device Name (padded with 00)} 
    \end{rightwordgroup} \\
    \bitboxes*{1}{{01} {02} {00} {36}} &
    \bitbox{1}{\emph{D}} & \bitbox{1}{01} &
    \bitbox{6}{MAC address} & \bitbox{4}{IP address} \\
    \bitboxes*{1}{{01} {00} {00} {00} {01} {00}} \\
  \end{bytefield}
  \caption{CDJ keep-alive packets}
  \label{fig:cdjKeepalive}
\end{figure}

As seems to always be the case when comparing mixer and CDJ packets,
the difference between this and Figure \ref{fig:mixerKeepalive} is
that byte 37 (following the device number \emph{D}) has the value 01
rather than 02, and the same is true of the second-to-last byte in
each of the packets. (Byte 52 is 01 in Figure \ref{fig:cdjKeepalive}
and 02 in Figure \ref{fig:mixerKeepalive}.

\end{document}

